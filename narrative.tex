\documentclass[twoside, leqno, twocolumn]{article}
\usepackage{ltexpprt}


\usepackage{fullpage} %[cm]
\usepackage{lmodern} % enhanced version of computer modern
\usepackage[T1]{fontenc} % for hyphenated characters
 \usepackage{amsmath}
 \usepackage{amssymb}
\usepackage{microtype}
\usepackage{enumerate}
\usepackage{ctable} % provides toprule, bottomrule, midrule
%\usepackage[ruled,linesnumbered]{algorithm2e}
% \usepackage{amsthm}
\usepackage{graphicx}
\usepackage{subfig}
\usepackage{breqn}
\DeclareCaptionType{copyrightbox}
\usepackage{color}
\newcommand{\expect}{\mathrm{Exp}}
\newcommand{\hide}[1]{}
%\newtheorem{theorem}{Theorem} % {\bfseries}{\itshape}
%\newtheorem{lemma}{Lemma}[section]{\bfseries}{\itshape}
% \theoremstyle{definition}
\newtheorem{definition}{Definition}[section]
%\newdef{claim}{Claim}
%
%\theoremstyle{definition}
%\newtheorem{definition}[lemma]{Definition} % {\bfseries}{\itshape}
%
%\theoremstyle{remark}
\newtheorem{claim}[lemma]{Claim} % {\bfseries}{\itshape}
\newtheorem{observation}[lemma]{Observation} % {\bfseries}{\itshape}
%
\DeclareMathOperator*{\argmin}{arg\,min}
\newcommand{\opt}{\mathrm{OPT}}
\newcommand{\eopt}{E_{\mathrm{OPT}}}
\newcommand{\edeg}{e_{\mathrm{deg}}}
\newcommand{\prob}{\textsc{Spectral Radius Minimization}}
\newcommand{\degr}{\mathrm{d}}
\newcommand{\Vol}{\mathrm{Vol}}
\newcommand{\wmax}{w_{\max}}
\newcommand{\eps}{\epsilon}
\newcommand{\diff}{\textsc{diff}}
\newcommand{\intS}{\textsc{Int}}
\newcommand{\cost}{\textsc{cost}}
\newcommand{\elt}{\text{elt}}
\newcommand{\ceil}[1]{\left\lceil #1 \right\rceil}
\newcommand{\red}[1]{\textcolor{red}{#1}}
\DeclareMathOperator*{\nodes}{nodes}
\DeclareMathOperator*{\walks}{walks}
\DeclareMathOperator*{\ct}{count}

\title{Set Summarization Problem}
%\author{}
%\date{}
%
\begin{document}
%\maketitle
%

%\section{Introduction}

\noindent
\textbf{Abstract:}~
 The problem of concisely summarizing datasets arises in many
applications including databases, computational biology, epidemiology
and business. One version of this problem where the goal is to
express a given target set  as the union of a small collection of
given sets has been studied in the literature.  We consider an
extension of the problem where the goal is to express a target set
as the set difference between the unions of two collections of sets.
We observe that allowing the difference operator can significantly
reduce the cost of summarization. Since the problem is NP-hard, we
develop approximation algorithms provable performance guarantees
for the general problem and some restricted versions. Our main
results exploit the fact that problem corresponds to minimizing a
submodular objective function subject to a constraint on another
submodular function. We also carry out an experimental evaluation
of the performance of our algorithms using public domain datasets.

\noindent
\textbf{Keywords:}~ Set summarization, optimization, approximation algorithms, submodular functions

\input{introduction.tex}
\input{relatedwork.tex}
\input{preliminaries.tex}
\input{methods.tex}
\input{results.tex}
\input{conclusions.tex}

\section{Using a submodular maximization approach for SSP}
We consider $k=1$.
Let $\mathcal{S} = \{S_1,\ldots,S_m\}$ denote the $m$ sets, which are subsets of
a ground set $U$. $T\subseteq U$ is a given target set.  For a subset
$X\subseteq \mathcal{S}$, let $u(X)=\bigcup_{S\in X} S$ denote the union of all the
sets in $X$.  The SSP problem is to
find sets $X=\{S_{i_1},\ldots,S_{i_r}\}$ and $Y=\{S_{j_1},\ldots,S_{j_s}\}$ such that
\[
T= u(X) - u(Y)
\]
The cost of such a representation is $|X|+|Y|$.

Let $\mathcal{S}_T=\{S\in\mathcal{S}: S\cap T=\emptyset\}$.
For a set $Z\subseteq U-T$, let $h(Z)$ denote the minimum number of sets from $\mathcal{S}_T$
needed to cover $Z$. We will assume $\mathcal{S}$ contains all singleton sets, so that
$h(Z)$ is always well defined.
Define functions $f,g:2^{\mathcal{S}}\rightarrow \mathbb{R}_{\geq 0}$ in the following manner.
For $X\subset \mathcal{S}$, $f(X)=|X| + h(u(X)-T)$ and $g(X)=|u(X)\cap T|$.

\begin{claim}
Suppose $X\subseteq\mathcal{S}$ such that $g(X)=|T|$. Then, there exists a solution $(X, Y)$
to the SSP problem with cost equal to $f(X)$. The converse also holds.
\end{claim}
\begin{proof}
Since $g(X)=|T|$, we have $T\subseteq u(X)$. Let $Y\subseteq \mathcal{S}_T$
such that $u(X)-T \subseteq u(Y)$, and $|Y|$ is minimized.
Such a $Y$ exists, and $|Y|=h(u(X)-T)$. Also, by construction, $u(Y)\cap T=\emptyset$.
Therefore,
\[
T = u(X) - u(Y),
\]
and $|X|+|Y| = f(X)$.
\end{proof}

\section{Preliminaries and problem formulation}

\subsection{Notation}
Let $U=\{x_1,\ldots,x_n\}$ denote a ground set, and let 
$\mathcal{S}=\{S_1,\ldots,S_m\}\subset 2^U$ denote a set of subsets of $U$. 
We assume that $\{x_1\},\ldots,\{x_n\}\in\mathcal{S}$, i.e., all the singleton
sets are in the set system.
Let $C(i_1,\ldots,i_k) = S_{i_1}\cap S_{i_2}\ldots\cap S_{i_k}$, where
each $S_{i_j}\in \mathcal{S}$. We refer to $C(i_1,\ldots,i_k)$ as a ``clause''.
Let $\intS_k=\{C(i_1,\ldots,i_{k'}): i_j\leq m, k'\leq k\}$
denote the set of all intersections of at most $k$ sets from $\mathcal{S}$.
We consider representations of a subset $T\subseteq U$ of the following form:
\[
T = \cup_{\ell=1}^r C(i^{\ell}_1,\ldots,i^{\ell}_{k_{\ell}}) - 
 \cup_{\ell=r+1}^s C(i^{\ell}_1,\ldots,i^{\ell}_{k_{\ell}}),
\]
where $C(i^{\ell}_1,\ldots,i^{\ell}_{k_{\ell}})\in \intS_k$.
We say the clauses $C(i^{\ell}_1,\ldots,i^{\ell}_{k_{\ell}})$ for $\ell\leq r$
are ``positive'', whereas, those for $\ell=r+1,\ldots,s$ are ``negative''.
For $C=C(i_1,\ldots,i_{k'})\in \intS_k$, we define $\cost(C)=k'$ as the
number of sets whose intersection defines $C$.
For a representation of $T$ of the above type, we associate
a cost $\alpha\sum_{\ell=1}^r k_{\ell} + \beta \sum_{\ell=r+1}^s k_{\ell}$,
where $\alpha$ and $\beta$ are parameters. 

\noindent
\textbf{Exact and approximate representations.}
The representation of the form above, where $T$ is equal to the difference of
unions of clauses, is said to be an \emph{exact} representation.
We consider the natural \emph{approximate} representations, where 
$T$ is ``close'' to $\cup_{\ell=1}^r C(i^{\ell}_1,\ldots,i^{\ell}_{k_{\ell}}) -
 \cup_{\ell=r+1}^s C(i^{\ell}_1,\ldots,i^{\ell}_{k_{\ell}})$.
\smallskip

\subsection{Problems}
\noindent
\textbf{1. Set summarization problem} (SSP). Given an instance $(U, \mathcal{S}, T, k)$,
the objective is to find a representation
\[
T = \cup_{\ell=1}^r C(i^{\ell}_1,\ldots,i^{\ell}_{k_{\ell}}) - 
 \cup_{\ell=r+1}^s C(i^{\ell}_1,\ldots,i^{\ell}_{k_{\ell}}),
\]
such that the cost 
$\alpha\sum_{\ell=1}^r k_{\ell} + \beta \sum_{\ell=r+1}^s k_{\ell}$ is minimized.

\noindent
\textbf{2. Positive version of the Set summarization problem} (positive-SSP). 
In this restriction, we consider representations consisting of only positive clauses, i.e.,
\[
T = \cup_{\ell=1}^r C(i^{\ell}_1,\ldots,i^{\ell}_{k_{\ell}}),
\]
so that the cost $\alpha\sum_{\ell=1}^r k_{\ell}$ is minimized.


\noindent
\textbf{3. Approximate version of the Set summarization problem
with false negatives} (approx-SSP-fn). 
In this version, the objective is to find a representation for a set $T'\subseteq T$ such that
\[
T' = \cup_{\ell=1}^r C(i^{\ell}_1,\ldots,i^{\ell}_{k_{\ell}}) - 
 \cup_{\ell=r+1}^s C(i^{\ell}_1,\ldots,i^{\ell}_{k_{\ell}}),
\]
such that $|T'|\geq (1-\epsilon)|T|$, and $\cost(T')$ is minimized.

\noindent
\textbf{4. Approximate version of the Set summarization problem
with false positives and false negatives} (approx-SSP). 

\subsection{More general formulation}

We have $p$ groups of sets $\mathcal{S}_j=\{S_{j1},\ldots,S_{jm_j}\}$, for $j=1,\ldots,p$.
Let $\mathcal{R}_j=\{R_{ji_1},\ldots,R_{ji_j}\}\subseteq\mathcal{S}_j$ 
be a subset of the $j$th group. Let $\elt(\mathcal{R}_j) = \cup_{R\in \mathcal{R}_j} R$
denote the union of the sets in $\mathcal{R}_j$.
A hyperset is denoted by $\mathcal{R}=(\mathcal{R}_1,\ldots,\mathcal{R}_p)$,
and consists of the following elements
\[
\cap_{j=1}^p \elt(\mathcal{R}_j)
\]

The cost $\cost(\mathcal{R})$ of such a hyperset is taken to be
\[
\cost(\mathcal{R}) = \sum_{j=1}^p |\mathcal{R}_j| = \sum_{j=1}^p i_j
\]

We consider representations of a (target) subset $T\subseteq U$ of the following form:
\[
T = \cup_{\ell=1}^r \mathcal{R}^{\ell} - \cup_{\ell=r+1}^s \mathcal{R}^{\ell},
\]
i.e., a difference of unions of hypersets. The cost of such a representation is
\[
\sum_{\ell=1}^s\cost(\mathcal{R}^{\ell})
\]

\noindent
\textbf{Generalization of Set Cover.}
The set cover problem is a special case of the above formulation, with a single group
$\mathcal{S}=\mathcal{S}_1$, so that a hyperset is an individual set $R\in\mathcal{S}$.

\noindent
\textbf{Generalization of \cite{xiang:dmkd2011}.}
The formulation above generalizes the 2-dimensional setting considered by
\cite{xiang:dmkd2011} in the following manner. The input consists of a set
$\mathcal{T}$ of $n_t$ transactions and a set $\mathcal{I}$ of $n_i$ items.
For $T\subset\mathcal{T}$ and $I\subset\mathcal{I}$,
a hyper-rectangle $H=T\times I=\{(i, j): i\in T, j\in I\}$.

Consider a universe $U=\{(i, j): i\in \mathcal{T}, j\in\mathcal{I}\}$ of $n_tn_i$ elements.
We have two groups of sets:
(1) $\mathcal{S}_1$ has a set $\{(i, j): j\in \mathcal{I}\}$ for each transaction 
$i\in\mathcal{T}$, and
(2) $\mathcal{S}_2$ has a set $\{(i, j): i\in \mathcal{T}\}$ for each item $i\in\mathcal{I}$.

\section{Related work}

The work of Xiang et al. \cite{xiang:dmkd2011} is directly related, and can be
considered as a special case of positive-SSP. The data comes from a transactional database,
where we have sets $\mathcal{I}$ and $\mathcal{T}$ of items and transactions, respectively.
Let $n_I=|\mathcal{I}|$ and $n_T=|\mathcal{T}|$. Then, the database
can be represented as a 0/1 matrix of dimensions $n_I\times n_T$.
The ground set $U$ is the set of all the matrix elements.
The set system $\mathcal{S}$ consists of all sets of the form
$T=\{i_1,\ldots,i_r\}\times\{t_1,\ldots,t_s\}$, where $\{i_1,\ldots,i_r\}\subset\mathcal{I}$
and $\{t_1,\ldots,t_s\}\subset\mathcal{T}$; these are referred to as 
``hyperrectangles'', and $\cost(T)=r+s$.
Xiang et al. give a logarithmic approximation for the positive-SSP problem
for such instances.

\section{Questions and results}

\begin{enumerate}
\item
SSP vs positive-SSP: for the 2D case \cite{xiang:dmkd2011}, 
we can construct a set $S$ corresponding to a rectangle of 1's with a single 0,
such that the optimum solution of positive-SSP version is close to three times
the optimum solution of the SSP version.
Can this gap be made larger for general SSP instances?
\item
The set system in \cite{xiang:dmkd2011} has bounded VC-dimension for ``most'' sets. 
Can we use this to improve the logarithmic approximation factor?
\item
Stronger hardness for SSP, compared to positive-SSP
\item
Logarithmic approximation for positive-SSP, at least when $k$ is bounded.
\end{enumerate}

\section{Approximation results}

\subsection{Set cover version}

We consider the approximation algorithm for the single group case, which 
is the set cover generalization. We use a simple LP rounding approach.

Let $\mathcal{S} = \{S_1,\ldots,S_m\}$ denote the $m$ sets. We have variables
$x(S)$ and $y(S)$ corresponding to the set $S$ being used in a positive and negative 
sense, respectively. We have the following LP

\begin{eqnarray*}
\min \sum_S x(S) + y(S) && \text{s.t.}\\
\forall i\in T,\ \sum_{S\ni i} x(S) &\geq& 1\\
\forall i\not\in T, S\ni i,\ \sum_{S'\ni i} y(S') &\geq& x(S)\\
\forall i\in T, S\ni i,\ y(S) &=& 0\\
\forall S,\ x(S), y(S) &\geq& 0
\end{eqnarray*}

Let $f=\max_i |\{S: i\in S\}|$ be the maximum number of sets containing any element.
Our algorithm rounds the fractional variables $x(S), y(S)$ to integral variables
$X(S), Y(S)$ in the following steps.
\begin{enumerate}
\item
For each $S$, if $x(S)\geq 1/f$, round it to 1, i.e., set 
variable $X(S)=1$. Otherwise, $X(S)=0$.
\item
For each $S$, round $y(S)$ to 1 with probability 
$\min\{1, 16f{\epsilon^2}\log{mn}\cdot y(S)\}$.
\end{enumerate}

\begin{lemma}
The rounded integral solution $X(\cdot), Y(\cdot)$ is feasible and has
cost at most $O(f\log{m} OPT)$, where $OPT$ is the cost of the optimal solution,
with probability at least $1-\frac{1}{mn}$.
\end{lemma}

\begin{proof}
For each $i\in T$, there exists at least one $S\ni i$ such that $x(S)\geq 1/f$,
by definition of $f$. Therefore, by our rounding step, 
for each $i\in T$, we have $\sum_{S\ni i} X(S)\geq 1$, and
$X(S)\leq f\cdot x(S)$.

Next, consider any $i\not\in T$ and $S\ni i$. We have
$\sum_{S'\ni i} fy(S') \geq X(S)$. We only consider constraints with $X(S)=1$; 
else, the constraint is automatically satisfied after the rounding.
Let $y'(S') = \min\{1, 16f{\epsilon^2}\log{mn}\cdot y(S)\}$.
Then, we have $\sum_{S'\ni i} y'(S') \geq 16\epsilon^2\log{mn}$.
$Y(S')$ is a random variable which is 1 with probability $y'(S')$. 
We ignore any constraint which has $y'(S')=1$ for some $S'$, since $Y(S')=1$,
and such a constraint would be automatically satisfied.

Therefore,
$E[Y(S')] = y'(S')$, and $E[\sum_{S'\ni i} Y(S')] \geq 16\epsilon^2\log{mn}$.

It follows by randomized rounding, that if
we round $y(S')$ to 1 with probability $\min\{1, cf\log{m} y(S')\}$, then with high
probability, we have $\sum_{S'\ni i} Y(S')\geq X(S)$
\end{proof}

\newpage

\section{Submodularity}

Let $S = \{S_1, S_2, ..., S_m\}$ be a collection of subsets of the ground set $U = \{x_1,  ..., x_n \}$.  Let $T$ be a target set such that $T \subseteq U$. Let $X, Y$ be subsets of $S$.

\begin{definition}
Let $S_T$ be a collection of subsets in $S$ such that 
\begin{equation}
	S_T = \{ S_i \in S : S_i \cap T = \phi \} \nonumber
\end{equation}
i.e., it consists of all subsets of $S$ that are disjoint with $T$
\end{definition}
\begin{definition}
 Let $v(X)$ be the union of all subsets in $X$ i.e.,
 \begin{equation}
 		v(X) = \cup_{S_i \in X} \; S_i  \nonumber
 \end{equation}
\end{definition}
\begin{definition}
	For any $T \subseteq U$, let  $cov(T)$ be defined as the minimum number of sets from $S_T$ needed to cover all elements of $T$, i.e.,
	\begin{equation}
		 	cov(T) = \min_{Z: Z \subseteq S, T- v(Z) = \phi} \; |Z|
	\end{equation}
\end{definition}
\begin{definition}
	Let $f \colon 2^S \to \mathbb{Z}$ be a function defined as $f(X) = |X| + cov( v(X) - T)$, i.e., sum of the number of subsets in $X$ and the minimum number of subsets needed to cover all elements in $v(X) - T$.
\end{definition}
\begin{lemma}
	The function $f$ is submodular.
\end{lemma}

\begin{definition}
	Let $g \colon 2^S \to \mathbb{Z}$ be a function defined as $g(X) = | v(X) \cap T|$, i.e, number of elements of $T$ covered by elements in $X$. 
\end{definition}

\begin{lemma}
	The function $g$ is submodular.
\end{lemma}
\begin{proof}

\end{proof}
\newpage
 
\bibliographystyle{plain}
\bibliography{references}

\end{document}

 
